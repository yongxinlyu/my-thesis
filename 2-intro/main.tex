\chapter{Introduction}\label{c:introduction}

\section{Background}
Recent advances in artificial intelligence (AI) have revolutionized materials discovery, enabling researchers to explore vast chemical and structural spaces with unprecedented efficiency. Traditional experimental and computational methods for materials design are often time-consuming and resource-intensive, making AI-driven approaches particularly attractive. By leveraging machine learning (ML) techniques, researchers can predict material properties, optimize design parameters, and identify promising candidates for various applications\cite{RN421}. Among these AI-driven strategies, inverse design has emerged as a transformative approach that reverses the conventional forward design process, allowing the direct discovery of materials with targeted properties\cite{RN361}. This approach employs generative models, optimization algorithms, and invertible material representations to streamline the discovery process across diverse material domains, including inorganic crystals, high-entropy alloys, organic semiconductors, and metal-organic frameworks\cite{RN412,RN354,RN612}.

Two-dimensional (2D) hybrid perovskites represent an exciting frontier for AI-assisted inverse design due to their structural tunability and unique optoelectronic properties. Compared to their three-dimensional (3D) counterparts, 2D perovskites offer a significantly larger design space owing to the incorporation of organic cation spacers. Among the various 2D perovskite phases, the Dion-Jacobson (DJ) phase has attracted significant attention for its distinctive structural features, including the presence of diammonium organic spacers and the absence of van der Waals gaps. These properties contribute to enhanced charge transport and stability, making DJ-phase perovskites particularly promising for optoelectronic applications such as photovoltaics and light-emitting diodes (LEDs). However, the rational design of organic spacers in DJ-phase perovskites remains a major challenge due to the vast chemical space and the complex interplay between molecular structure and electronic properties\cite{RN144}.

Despite recent progress in AI-assisted workflows for hybrid perovskites, most studies have focused on forward design approaches that rely on exhaustive searches within predefined chemical spaces\cite{RN315,RN283}. These methods often prioritize formability and stability while overlooking critical properties such as energy level alignment, which directly influences charge transport and device performance. Given the quantum-well-like structure of 2D perovskites, where organic and inorganic layers possess distinct electronic properties, understanding and optimizing energy level alignment is crucial for advancing these materials in optoelectronic applications\cite{RN18}. Addressing this challenge requires a systematic and AI-driven inverse design methodology tailored to hybrid perovskites.


\section{Research Objectives}

The primary objective of this research is to develop an AI-assisted inverse design framework for discovering new organic spacers in DJ-phase hybrid perovskites, with a particular focus on optimizing energy level alignment. The specific goals of this study are:
\begin{itemize}
\item To explore AI-driven approaches for materials discovery and evaluate their applicability to 2D hybrid perovskites.
\item To investigate inverse design methodologies for identifying organic spacers in 2D perovskites with tailored electronic properties.
\item To examine the impact of organic spacers on the electronic structure and synthesize feasibility of DJ-phase hybrid perovskites.
\item To inverse design new organic spacer candidates to achieve the targeted energy level alignment of DJ perovskites
\end{itemize}

This research aims to bridge the gap between AI-assisted design and hybrid perovskite discovery, providing a systematic approach for accelerating materials innovation through inverse design principles.

\section{Thesis Structure}
This thesis is structured into six chapters as follows:

Chapter \ref{c:introduction} (Introduction) sets the context for AI-assisted materials discovery, emphasizing the importance of inverse design within hybrid perovskites. It also states the central research objectives and outlines the scope of this work.

Chapter \ref{c:literature} (Literature Review) surveys recent advances in AI-driven materials research and explores various inverse design methodologies. It then examines the structural fundamentals of 2D perovskites—particularly Dion–Jacobson (DJ) perovskites—and highlights why they provide a compelling platform for investigating organic spacer design. Finally, it reviews current AI techniques used for 2D perovskite discovery and underscores the value of pursuing inverse design strategies in this domain.

Chapter \ref{c:method} (Methodology) presents the overall inverse design framework. It details the development of invertible molecular fingerprints, molecular morphing, high-throughput calculations, machine learning models for energy level prediction, and synthesis feasibility screening.

Chapter \ref{c:result-1} (High-Throughput Calculation and Machine Learning Predictions) details the expansion of the chemical space through molecular morphing, visualization of generated structures, ML model selection and evaluation, and interpretation of structure-property relationships.

Chapter \ref{c:result-2} (Synthesis Feasibility Screening and Final Candidate Validation) explains the screening protocol for synthetic accessibility and examines the structural formability of 2D layers. It also focuses on identifying the fingerprint features that correlate with specific energy alignment types and concludes with DFT-based validation of the most promising DJ-phase perovskite candidates.

Chapter \ref{c:conclusion} (Summary and Outlook) summarizes the key findings of the research and provides an outlook on future directions for AI-assisted hybrid perovskite design.









