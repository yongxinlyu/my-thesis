\chapter{Acknowledgement}

Looking back on my PhD journey, I feel incredibly lucky—not just for the experience itself, but for all the amazing people I met along the way and those who helped me get here in the first place.

First and foremost, I would like to express my deepest gratitude to Prof. Tom Wu, my primary supervisor. Thank you for all the time and effort you put into my project—every revision, every piece of feedback, and all the collaborations you helped arrange. This project wouldn’t have been possible without you. One of the biggest lessons I’ve learned from you is the importance of critical thinking—questioning research, challenging assumptions, and pushing for a deeper understanding. I hope to carry that mindset with me throughout my career.

I also want to thank Prof. Jianhua Hao, my MPhil supervisor, especially your support during the beginning of my PhD when I was working remotely. And a special shoutout to Prof. Ran Ding, my former MPhil group member—you were the one who first got me interested in perovskite research, which ultimately led me here.

To my research group members and everyone I met in the School of Materials Science and Engineering and School of Chemistry, thank you for your kindness, encouragement, and all the little moments of support along the way. A special thanks to Alan, for sharing your valuable insights with me. And to the 2024 PGSOC members—thanks for welcoming me into the community.

A massive thank you to Prof. Mira Kim—a mentor, a friend, and the person who introduced me to the PELE community. Your encouragement has continuously pushed me to step outside my comfort zone, and your constant support has meant the world to me. To all the tutors, mentors, and wonderful friends I met through PELE, thank you for making this journey even more meaningful.

Keeping some sense of balance throughout this PhD was crucial, and for that, I have to thank the yoga and Pilates instructors at the UNSW Fitness Centre, as well as the wonderful friends I made there. Yoga became a huge part of my life, giving me some much-needed clarity and calm during this rollercoaster of PhD journey.

On the technical side, I want to acknowledge the national supercomputers, Gadi and Setonix, along with ResTech at UNSW for providing computing resources, responsive support, and training sessions that helped me sharpen my coding skills.

I would also like to acknowledge the Australian Government Research Training Program (RTP) Scholarship for its financial support, which made this PhD journey possible.

Finally, I want to express my heartfelt gratitude to my family. To my parents, thank you for always supporting my dreams, even though we are separated by thousands of miles—I miss you deeply. And to my cousin, Zefang, I am grateful for our shared reflections on the struggles of PhD life—wishing you all the best in your own journey.

To my husband, Sa, who has tirelessly (but unsuccessfully) tried to convince me that chemistry is superior to materials science. Sharing this PhD journey with you has meant celebrating each other’s victories, navigating the struggles together, and always having a teammate through the ups and downs. Thank you for the countless afternoon coffee runs, the nerdy scientific debates, and for making this experience not just bearable, but truly enjoyable. Most of all, thank you for being my rock, my study buddy, and my greatest supporter every step of the way.
